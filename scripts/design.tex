\chapter{Design}
Discuss general design alternatives. Give equations, simulations, general circuits. Describe design in detail, addressing each major component. Include schematics with components, drawings, flowcharts, etc. Some teams may wish to split this chapter in two: 2. Design Procedure, and 3. Design Details. This template will not automatically update numbering systems for chapters, sections, figures, tables, etc., so keep track of them as you develop and revise the text.

\section{Component or Block}
To create a section head, use \verb|\section{}|. It automatically formats as above and creates a table of contents entry (after you compile).

\subsection{Subcomponent or subblock}
To create a subsection head, use \verb|\subsection{}|. It automatically formats as above and creates a table of contents entry.

Following is a “template” for displayed math. It looks much better than the Word template.

\begin{align}
    EQO            & = \sum_{i=1}^n W_j * r_{ij}                                           \\
    EHI            & = L_1 \times ESI + L_2 \times EQI                                     \\
    \frac{EE}{EHI} & = \beta_0 + \beta_1 PCG + \beta_2 RGP + \cdots + \beta_i X_i +        \\
                   & \cdots + \beta_9 ICWUR + \beta_{10} ECPG + \beta_11 WCPG + \epsilon_i
\end{align}

\section{Theorems}
Theorems can easily be defined:

\begin{thm}
    Let \(f\) be a function whose \textbf{derivative} exists in every point, then \(f\) is
    a continuous function.
\end{thm}
\begin{proof}
    To prove it by contradiction try and assume that the statement is false,
    proceed from there and at some point you will arrive to a contradiction.
\end{proof}

\begin{thm}[Pythagorean theorem]
    \label{pythagorean}
    This is a theorem about right triangles and can be summarised in the next
    equation
    \[ x^2 + y^2 = z^2 \]
\end{thm}

And a consequence of theorem \ref{pythagorean} is the statement in the next
corollary.

\begin{cor}
    There's no right rectangle whose sides measure 3cm, 4cm, and 6cm.
\end{cor}

You can reference theorems such as \ref{pythagorean} when a label is assigned.

\begin{lem}
    Given two line segments whose lengths are \(a\) and \(b\) respectively there is a
    real number \(r\) such that \(b=ra\).
\end{lem}